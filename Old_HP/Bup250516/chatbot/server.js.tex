// dotenvパッケージを使って環境変数を読み込む
require('dotenv').config();

// Supabaseパッケージをインポート
const { createClient } = require('@supabase/supabase-js');

// 環境変数からSupabaseのURLとAPIキーを取得
const SUPABASE_URL = process.env.SUPABASE_URL;
const SUPABASE_KEY = process.env.SUPABASE_KEY;

// Supabaseクライアントを作成
const supabase = createClient(SUPABASE_URL, SUPABASE_KEY);

// Expressを使ってサーバーを作成
const express = require('express');
const app = express();
app.use(express.json());

// チャットデータをSupabaseに保存するエンドポイント
app.post('/chat', async (req, res) => {
  const { message } = req.body;
  const { data, error } = await supabase
    .from('conversation_logs')
    .insert([{ user_message: message, bot_response: 'Sample response', created_at: new Date() }]);

  if (error) {
    return res.status(500).json({ error: error.message });
  }
  res.status(200).json({ message: 'Data saved successfully', data });
});

require('dotenv').config();  // .envファイルを読み込む

console.log('SUPABASE_URL:', process.env.SUPABASE_URL);  // 環境変数を表示
console.log('SUPABASE_KEY:', process.env.SUPABASE_KEY);

async function getData() {
    const { data, error } = await supabase
        .from('your-table')
        .select('*');
    
    if (error) {
        console.log('Error:', error);
    } else {
        console.log('Data:', data);
    }
}

getData();


// サーバーを起動
app.listen(3000, () => {
  console.log('Server is running on http://localhost:3000');
});
